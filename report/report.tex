\documentclass{article} 
\usepackage{polyglossia} 
\usepackage{amsmath}
\usepackage{fontspec} 
\usepackage{lipsum} 
\usepackage[margin=1in]{geometry}
\usepackage{graphicx} 
\usepackage{caption} 
\usepackage{subcaption}
\usepackage{hyperref} 
\hypersetup{% 
    colorlinks=true, linkcolor=blue, filecolor=magenta,      
    urlcolor=cyan, 
    pdfinfo = {% Title = Συστήματα
        Πολυμέσων Υλοποίηση GSM 06.10 Author = Χρήστος Μάριος Περδίκης, Γιώργος
        Β.  Producer = XeLaTeX 
    } 
}

\title{Συστήματα Πολυμέσων Υλοποίηση GSM 06.10} 
\author{Χρήστος Μάριος Περδίκης 10075 cperdikis@ece.auth.gr 
\and Γιώργος RootCritter ?????  ?????????@ece.auth.gr} 
\date{}

\setmainfont{CMU Serif}

\begin{document} \maketitle Το παρόν έγγραφο είναι η αναφορά της
εργασίας του μαθήματος Συστήματα Πολυμέσων τη χρονιά 2024-2025.
Κληθήκαμε να υλοποιήσουμε κωδικοποίηση φωνής σύμφωνα με το πρότυπο GSM
06.10. Από τα τρία επίπεδα υλοποιήσαμε μέχρι το 1o.

\section{Προετοιμασία}
Το πρόγραμμά μας τρέχει με την εντολή ``\verb|python main.py|''.
Στο αρχείο \verb|main.py| υπάρχει η λογική δομή του προγράμματος,
από εκεί καλούνται όλες τις συναρτήσεις που υλοποιούν τις λειτουργίες 
του κωδικοποιητή. Θα περιγραφεί πρώτα η γενική λογική του προγράμματος 
και έπειτα θα εξηγηθεί λεπτομερώς κάθε επιμέρους τμήμα. Αρχικά
δίνεται ένα filename ενός wav αρχείου σε ένα audio wrapper, το οποίο
μετατρέπει τα samples του ηχητικού αρχείου σε μορφή στην οποία
μπορεί να γίνει επεξεργασία.  Τα samples υπόκεινται σε μια
διαδικασία pre-processing και ακολουθεί το στάδιο 
short-term analysis. Το σήμα φωνής εισέρχεται πρώτα σε έναν κωδικοποιητή 
και μετά σε έναν αποκωδικοποιητή. Τέλος, τα samples που προέκυψαν μετά την
αποκωδικοποίηση γράφονται σε ένα νέο wav αρχείο μέσω του audio wrapper.
Με σύγκριση του αρχικού αρχείου φωνής και του αρχείου φωνής που 
δημιούργησε το πρόγραμμά μας, μπορούμε εύκολα να παρατηρήσουμε τα αποτελέσματα 
της κωδικοποίησης. 

\subsection{Audio Wrapper} 
Οι λειτουργίες του audio wrapper υλοποιούνται στο αρχείο 
\verb|audio_wrapper.py|. Από το \verb|main.py| καλείται η συνάρτηση 
\verb|scipy_read_data| με όρισμα ένα filename string. Με την αλλαγή 
αυτού του string είναι δυνατή η τροφοδότηση ενός διαφορετικού αρχείου
ήχου στον κωδικοποιητή. Χρησιμοποιήσαμε το \verb|scipy| module για την 
ανάγνωση samples από το αρχείο ήχου, καθώς με την κλήση μιας συνάρτησης με
όρισμα μόνο το filename, μπορεί να διαχειριστεί wav αρχεία με διαφορετικό
αριθμό bits για κάθε sample, ανεξάρτητα με το αν είναι signed ή unsigned
και ανεξάρτητα του endianness. Αρχικά είχαμε επιχειρήσει να χρησιμοποιήσουμε
το \verb|wave| module, το οποίο όμως απαιτούσε manual input για όλες τις 
παραπάνω παραμέτρους.

(audio wrapper write function)

\subsection{Pre-processing}
Οι λειτουργίες του preprocessing υλοποιούνται στο αρχείο 
\verb|preprocessing.py|. Πρώτα καλείται η συνάρτηση \verb|offset_compensation|
η οποία υλοποιεί την διαδικασία του offset compensation όπως περιγράφεται στο 
πρότυπο. Η είσοδός της είναι τα samples φωνής \verb|s0|, δηλαδή η έξοδος του 
audio wrapper. Η έξοδός της είναι το τροποποιημένο array \verb|sof|, το οποίο 
με τη σειρά του είναι η είσοδος της συνάρτησης \verb|pre_emphasis|. Η τελευταία
υλοποιεί τη διαδικασία του pre-emphasis όπως περιγράφεται στο πρότυπο και έχει 
έξοδο το array με samples \verb|s|. Έτσι τελειώνει το pre-processing και είμαστε
έτοιμοι για το short-term analysis.


\section{Short Term Analysis} 
\subsection{Κωδικοποιητής}
\subsection{Αποκωδικοποιητής}

\end{document}
