\documentclass{article} 
\usepackage{polyglossia} 
\usepackage{amsmath}
\usepackage{fontspec} 
\usepackage{lipsum} 
\usepackage[margin=1in]{geometry}
\usepackage{graphicx} 
\usepackage{caption} 
\usepackage{subcaption}
\usepackage{hyperref} 
\hypersetup{% 
    colorlinks=true, linkcolor=blue, filecolor=magenta,      
    urlcolor=cyan, 
    pdfinfo = {% Title = Συστήματα
        Πολυμέσων Υλοποίηση GSM 06.10 Author = Χρήστος Μάριος Περδίκης, Γιώργος
        Β.  Producer = XeLaTeX 
    } 
}

\title{Συστήματα Πολυμέσων Υλοποίηση GSM 06.10} 
\author{Χρήστος Μάριος Περδίκης 10075 cperdikis@ece.auth.gr 
\and Γιώργος RootCritter ?????  ?????????@ece.auth.gr} 
\date{}

\setmainfont{CMU Serif}

\begin{document} \maketitle Το παρόν έγγραφο είναι η αναφορά της
εργασίας του μαθήματος Συστήματα Πολυμέσων τη χρονιά 2024-2025.
Κληθήκαμε να υλοποιήσουμε κωδικοποίηση φωνής σύμφωνα με το πρότυπο GSM
06.10. Από τα τρία επίπεδα υλοποιήσαμε μέχρι το 1o.

\section{Short Term Analysis} 
\subsection{Audio Wrapper και main.py} Αρχικά
δημιουργήσαμε ένα πολύ απλό audio wrapper πρόγραμμα ώστε να μπορούμε να
μετατρέπουμε τα ηχητικά δεδομένα ενός wav αρχείου σε μορφή την οποία
μπορεί να επεξεργαστεί από κώδικα, και μετά πάλι πίσω σε wav μορφή. Ο
κώδικας του προγράμματος αυτού βρίσκεται στο αρχείο
\emph{audio\_wrapper.py}. 

Υπάρχουν δύο συναρτήσεις για την ανάγνωση wav αρχείων. Αρχικά
κατασκευάσαμε τη συνάρτηση read\_data, η οποία έκανε χρήση του wave 
module και διαβάζαμε $160$ frames του αρχείου ήχου τη 
φορά. Παρουσιάστηκε
πρόβλημα όμως όταν θέλαμε να μετατρέψουμε το byte buffer που επιστρέφει
η συνάρτηση readframes σε numpy array. Συγκεκριμένα, έπρεπε να γνωρίζουμε
πόσα bit υπήρχαν σε κάθε sample, και δεν γνωρίζαμε αν έπρεπε να
λάβουμε υπόψιν το endianness των bytes και το αν ήταν signed ή unsigned
ακέραιοι. Για αυτό χρειάστηκε να κατασκευάσουμε τη συνάρτηση 
\emph{scipy\_read\_data}, η οποία έχει υλοποίηση με το module scipy. Η 
συνάρτηση io.wavfile.read οποία επιστρέφει απευθείας ένα
έτοιμο numpy array με τα samples ολόκληρου του αρχείου, λύνοντας τα 
προβλήματά μας. Κρατάμε και τις δύο συναρτήσεις για ιστορικούς λόγους, 
θεωρητικά δουλεύει και η read\_data για $8$-bits- και $16$-bits-per-sample
wav αρχεία, αλλά δοκιμάστηκε μόνο για να διαβάζει τα πρώτα $160$ samples
ενός wav αρχείου, όχι ολόκληρα αρχεία.

(να πω και για main.py εδώ; Καλό σημείο θα ήταν)

(audio wrapper write function)

\subsection{Κωδικοποιητής}
\subsection{Αποκωδικοποιητής}

\end{document}
